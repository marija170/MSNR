% !TEX encoding = UTF-8 Unicode
\documentclass[a4paper]{article}

\usepackage{color}
\usepackage{url}
\usepackage[T2A]{fontenc} % enable Cyrillic fonts
\usepackage[utf8]{inputenc} % make weird characters work
\usepackage{graphicx}

\usepackage[english,serbian]{babel}
%\usepackage[english,serbianc]{babel} %ukljuciti babel sa ovim opcijama, umesto gornjim, ukoliko se koristi cirilica

\usepackage[unicode]{hyperref}
\hypersetup{colorlinks,citecolor=green,filecolor=green,linkcolor=blue,urlcolor=blue}

\usepackage{listings}

%\newtheorem{primer}{Пример}[section] %ćirilični primer
\newtheorem{primer}{Primer}[section]

\definecolor{mygreen}{rgb}{0,0.6,0}
\definecolor{mygray}{rgb}{0.5,0.5,0.5}
\definecolor{mymauve}{rgb}{0.58,0,0.82}

\lstset{ 
  backgroundcolor=\color{white},   % choose the background color; you must add \usepackage{color} or \usepackage{xcolor}; should come as last argument
  basicstyle=\scriptsize\ttfamily,        % the size of the fonts that are used for the code
  breakatwhitespace=false,         % sets if automatic breaks should only happen at whitespace
  breaklines=true,                 % sets automatic line breaking
  captionpos=b,                    % sets the caption-position to bottom
  commentstyle=\color{mygreen},    % comment style
  deletekeywords={...},            % if you want to delete keywords from the given language
  escapeinside={\%*}{*)},          % if you want to add LaTeX within your code
  extendedchars=true,              % lets you use non-ASCII characters; for 8-bits encodings only, does not work with UTF-8
  firstnumber=1000,                % start line enumeration with line 1000
  frame=single,	                   % adds a frame around the code
  keepspaces=true,                 % keeps spaces in text, useful for keeping indentation of code (possibly needs columns=flexible)
  keywordstyle=\color{blue},       % keyword style
  language=Python,                 % the language of the code
  morekeywords={*,...},            % if you want to add more keywords to the set
  numbers=left,                    % where to put the line-numbers; possible values are (none, left, right)
  numbersep=5pt,                   % how far the line-numbers are from the code
  numberstyle=\tiny\color{mygray}, % the style that is used for the line-numbers
  rulecolor=\color{black},         % if not set, the frame-color may be changed on line-breaks within not-black text (e.g. comments (green here))
  showspaces=false,                % show spaces everywhere adding particular underscores; it overrides 'showstringspaces'
  showstringspaces=false,          % underline spaces within strings only
  showtabs=false,                  % show tabs within strings adding particular underscores
  stepnumber=2,                    % the step between two line-numbers. If it's 1, each line will be numbered
  stringstyle=\color{mymauve},     % string literal style
  tabsize=2,	                   % sets default tabsize to 2 spaces
  title=\lstname                   % show the filename of files included with \lstinputlisting; also try caption instead of title
}

\begin{document}

\title{Naslov seminarskog rada\\ \small{Seminarski rad u okviru kursa\\Metodologija stručnog i naučnog rada\\ Matematički fakultet}}

\author{Prvi autor, drugi autor, treći autor, četvrti autor\\ kontakt email prvog, drugog, trećeg, četvrtog autora}

%\date{9.~april 2015.}

\maketitle

\abstract{
U ovom tekstu je ukratko prikazana osnovna forma seminarskog rada. Obratite pažnju da je pored ove .pdf datoteke, u prilogu i odgovarajuća .tex datoteka, kao i .bib datoteka korišćena za generisanje literature. Na prvoj strani seminarskog rada su naslov, apstrakt i sadržaj, i to sve mora da stane na prvu stranu! Kako bi Vaš seminarski zadovoljio standarde i očekivanja, koristite uputstva i materijale sa predavanja na temu pisanja seminarskih radova. Ovo je samo šablon koji se odnosi na fizički izgled seminarskog rada (šablon koji \emph{morate} da koristite!) kao i par tehničkih pomoćnih uputstava. Pročitajte tekst pažljivo jer on sadrži i važne informacije vezane za zahteve obima i karakteristika seminarskog rada.}

\tableofcontents

\newpage

\section{Uvod}
\label{sec:uvod}

Elixir(Eliksir) je programski jezik koji se prvi put pojavljuje u javnosti 2012. godine kao projekat kompanije Plataformatec. Ovaj funkcionalni, dinamičan programski jezik*(footnote: Diskusija o specificnoj paradigmi u daljem tekstu) se pokreće na Erlang vituelnoj mašini pa samim tim i deli pogodna svojstva kao što su konkurentnost i tolerisanje grešaka, koje dolaze sa ovim okruženjem. Njegov tvorac, Jose Valim, navodi da je motivacija za pravljenje ovog jezika upravo bila ljubav prema ovoj virtualnoj mašini i ekosistemu, ali i njeni nedostaci. Iz tačke gledišta olakšavanja svakodnevnog razvoja sofvera, koncepti poput metaprogramiranje - tehnika kojom programi imaju mogućnost da druge programe posmatraju kao svoje podatke i na taj način čitaju pa čak i modifikuju njihov, a samim tim i svoj kod u vreme izvršavanja, zatim polimorfizam (pretp. da čitaoc zna, pitati prof.) i makroi kao i podrška za alate, bili su neke od nepostojećih karakteristika ovog sistema koje bi Elixir trebalo da nadomesti. Cilj ovog rada je da čitaoca bliže upozna sa osobinama, funkcionalnostima i specifičnostima ovog jezika kao i da kroz uporedni prikaz sa jezicima koji dele slične kocepte ili imaju istu upotrebu u određenim domenima, prikaže make i prednosti Elixira.
% Kada budete predavali seminarski rad, imenujete datoteke tako da sadrže redni broj teme, temu seminarskog rada, kao i prezimena članova grupe. Precizna uputstva na temu imenovnja će biti data na formi za predaju seminarskog rada. Predaja seminarskih radova biće isključivo preko veb forme, a NE slanjem mejla. Link na formu će biti dat u okviru obaveštenja na strani kursa. Vodite računa da prilikom predavanja seminarskog rada predate samo one fajlove koji su neophodni za ponovno generisanje pdf datoteke. To znači da pomoćne fajlove, kao što su .log, .out, .blg, .toc, .aux i slično, \textbf{ne treba predavati}.

\section{Istorijat}
\label{sec:istorijat}

Tokom 1980ih, telekomunikaciona kompanija Ericsson ispitivala je problem konkurentnosti, i nakon izvršenih testiranja odlučila je da razvije svoj jezik koji bi bio zasnovan na ovoj funkcionalnosti: Erlang. U to vreme brojnost jezika nije bila velika, kao danas. Programski jezik C je bio najzastupljeniji, ali uprkos njegovoj prilagodljivosti, tri programera Joe Armstrong, Mike Williams i Robert Virding, koji su bili zaposleni u Ericsson-u, su shvatili da mogu biti produktivniji ukoliko koriste programski jezik višeg nivoa. Nakon eksperimenata sa postojećim jezicima, odlučili su se da ni jedan ne zadovoljava njihove potrebe u dovoljnoj meri, pa su tokom 1987. započeli pisanje Erlanga i predstavili ga tri godine kasnije.

Jezik je najviše korišćen od strane telekomunikacionih kompanija, a glavni fokus je bila distribucija, tolerantnost na otkaze u sistemu, i hot-swappinga koda (mogućnost izmene koda programa bez potrebe prekida izvršavanja). U 1998. Ericsson se odlučuje da kod programskoj jezika Erlang učini dostupan svima i proglašava ga otvorenim (open source). Od tada je jezik korišćen u veoma specifičnim projektima, kao što je dobro poznati broker za poruke RabbitMQ, XMPP server “ejabberd”, i Apache CouchDB, koji je jedna od najpoznatijih distribuiranih baza podataka. Neke kompanije koriste Erlang za telekomunikacione proizvode, a primer su WhatsApp i Riot Games - kompanija koja je napravila igru League of Legends.

Tokom 2010., Jose Valim, u to vreme zaposlen na poziciji programera u kompaniji Plataformatec, radio je na poboljšanju performansi Ruby on Rails framework-a na višejezgarnim sistemima i bio je sve više frustriran ovim poslom. Shvatio je da Ruby nije bio dovoljno dobro dizajniran da reši problem konkurentnosti, pa je započeo istraživanje drugih tehnologija koje bi bile prihvatljivije. Tako je otkrio je Erlang, i upravo ga je interesovanje prema Erlangovoj virtuelnoj mašini podstaklo da započne pisanje Elixira. Uticaj projekta na kome je do tada radio odrazio se na to da Elixir ima sintaksu koja je nalik na Ruby-jevu. Ovaj jezik se pokazao veoma dobro pri upravljanju milionima simultanih konekcija: u 2015. Phoenix - zabeleženo upravljanje 2 miliona WebSocket konekcija, dok je u 2017. za skalirani Elixir zabeležena obrada 5 miliona istovremenih korisnika. (Q: Mozda bismo mogli neki grafik ili sliku o ovome) Elixir se danas koristi u velikim kompanijama, kao što su Pinterest, Moz, a upotrebu nalazi čak i u bankarskim sistemima. (q: pitati jel medium relevantan)( ref: https://medium.com/margobank/why-elixir-546427542c)

\section{Erlang VM i OTP}
1. Mora da se navede da je elixir nastao iz erlanga
    - Pise u uvodu a i u istorijatu, ali mozes to da napises kao pocetak poglavlja. Tipa "Osim sto je Erlang kao motivacija uticao na pravljenje Elixira, moze se reci da je Elixir nastao iz Erlanga. On se izvrsava na Erlangovoj virtualnoj masini i prosiruje njegov skup funckionalnosti, pa je vazno opisati koncepte funkcionisanja ove VM." (Prosrao sam se carski, al kontas poentu hahahaha)

2. Da se kompajliranjem elixir prevodi u .beam fajl koji se moze izvrsiti u okviru erlang vm

3. Sta cini erlang vm tako dobrom za konkurentnost

u .tex fajlu se nalazi zakomentarisan materijal koji treba preraditi u smislenu celinu

poredjenje jvm i erts http://liyunzhen.blogspot.com/2017/03/tech-note-java-virtual-machinejvm-vs.html

iz knjige prepricati deo o erlang vm

% In the Erlang VM, all code runs in tiny concurrent processes, each with its
% own state. Processes talk to each other via messages. And since all communi-
% cation happens by message-passing, exchanging messages between different
% machines on the same network is handled transparently by the VM, making
% it a perfect environment for building distributed software!
% However, I felt there was still a gap in the Erlang ecosystem. I missed first-
% class support for some of the features I find necessary in my daily work, things
% such as metaprogramming, polymorphism, and first-class tooling. From this
% need, Elixir was born.

% In Erlang/Erlang virtual machine case, I was forced to think the relationship among OS processes, OS threads and CPU cores.  To make it simple, 

% 1. An OS process is a kind of resource center of execution units, which are called  threads. 

% 2. An OS process must contains at least one thread. Each thread is executed by one CPU core. (Normal applications normally contain several threads. For example, Microsoft office word uses at least two threads, one for data-saving, one for UI display. So that office word could auto-save the document while the user is editing.) 

% 3. Processes share nothing among themselves. OS processes shares no memory. Erlang processes shares no memory.

% 4. An Erlang related OS process will spawn an OS thread for each CPU core. Such OS thread is called scheduler, which could manage a large number of Erlang process.


\section{Osobine jezika}
U ovom poglavlju(q: da li su ovo naslovi poglavja) će biti opisane osobine Elixira, osnove njegove sintakse, semantike, kao i podrška za koncepte koji su odlike funkcionalnih i konkurentnih jezika. (q: mesanje vremena) 

Pre nego što započnemo priču o tipovima, par reči o \textbf{Kernelu}. To je podrazumevano okruženje koje se koristi u Elixiru. Ono sadrži primitive jezika kao što su: \textit{aritmetičke operacije}, rukovanje \textit{procesima} i \textit{tipovima}, \textit{makroe} za definisanje novih funkcionalnosti (\textit{funkcija, modula...}), provere \textit{guard-ova} - predefinisanog skupa funkcija i makroa koji proširuju mogućnost \textit{pattern matching}-a itd. Sve ove funkcionalnosti se mogu pozivati bez prefiksa \textit{Kernel} jer su podrazumevano importovane po pokretanju interpretera (ovo važi i ukoliko se program kompajlira). Ukoliko pak korisnik ne želi da uključi pojedine funkcije/makroe može to uraditi korisćenjem \textbf{:except} opcije \textit{import} funkcije. (t: primer)
\label{sec:osobine}
\subsection{Ugrađeni tipovi}
\label{sec:tipovi}
Elixir implementira desetine tipova. Od njih je važno istaći ugradjene - primitivne tipova, preko kojih su ostali definisani:
\begin{itemize}
  \item Atomi (\textit{Atom})
  \item Celi brojevi (\textit{Integer})
  \item Brojevi u pokretnom zarezu(\textit{Float})
  \item Procesi (\textit{Process})
  \item Portovi (\textit{Port})
  \item Uređene torke (\textit{Tuple})
  \item Liste (\textit{List})
  \item Mape (\textit{Map})
  \item Funkcije (\textit{Function})
  \item Niske bitova 
  \item Reference
\end{itemize}

U zagradama nakon tipa, osim u poslednja dva koji nemaju odgovarajuće module, navena su imena modula koji sadrže funckije koje se koriste za operacije nad tim tipom. Imena ovih modula ne treba mešati sa primitivnim tipovima navedenim gore, iako oni sami jesu tip. Oni se mogu zamisliti kao neka vrsta omotača oko primitivnog tipa koji obezbedjuje bogatije funkcionalnosti nad njime. 
\begin{primer}
Literal [...] može biti iskorišćen da se napravi lista (primitiva), nad njom je moguće iskoristiti operator | koji bi je dekomponovao na glavu i rep ili od nje napravio novu listu (detalji u predstojećim poglavljima). U modulu \textit{List} imamo funkciju \textit{last} koja kada se primeni na listu, kao rezultat vraća njen poslednji element.
\end{primer}

Može biti čudno što se na ovoj listi nisu našle niske ili strukture, ali one su deo složenih tipova podržanih od strane Elixira. Takodje, postoji debata o tome da li su regularni izrazi i opsezi (\textit{Ranges}) tipovi za sebe i u nekoj literaturi se posmatraju ovako iako su tehnički strukture. (ref: Programming Elixir 1.3)

\subsubsection{Atomi, celi i brojevi u pokretno zarezu i opsezi}
\label{sec:ime}
Po rečenica o svakom (q: Da li da imamo uopšte objasnjavanja o tipovima)
\subsubsection{Procesi, portovi, reference i niske bitova}
\label{sec:ime}
Po rečenica o svakom
\subsubsection{Torke}
\label{sec:ime}

\subsubsection{Liste}
\label{sec:ime}
(t: Mozda pre tipova immutabilnost i pattern matching zbog ovoga)
\subsubsection{Mape}
\label{sec:ime}

\subsection{Anonimne funkcije}
\label{sec:ime}

\section{Instalacija}
Tralalal

\section{Primer}
Lalalala

\section{Frameworks}
lalalal
\subsection{Opste}
lalalala
\subsection{Prvi}
Traaaalalalalal

\section{Napredni koncepti}

\section{Poredjenje}


%2. Elixir je dizajniran da bude proširiv, omogućavajući developerima da prirodno prošire jezik na određene domene, kako bi povećali svoju produktivnost. Danas se najvise koristi za web programiranje, ali ima udela i u projektima vezanim za robotiku.
% Pošto je Elixir izgrađen na makroima lako je pomisliti da ih svaka biblioteka zahteva, međutim to nije slučaj. Makroi treba da budu rezervisani za specijalizovane slučajeve gde rešenje ne može biti lako implementirano kao obične funkcije, jer se vrši generisanje koda. U nekim slučajevima, izbor makronaredbi je neophodan.Glavna odgovornost pisanja čistog koda sa makroima pada na programere. Makroe je teže pisati od običnih funkcija Elixir-a i smatra se lošim stilom da ih koristimo kada nisu neophodni. Elixir već pruža mehanizme za pisanje vašeg svakodnevnog koda na jednostavan i čitljiv način koristeći svoje strukture podataka i funkcije. Makroe treba koristiti samo kao krajnje sredstvo. 

%3. Možemo pisati kôd koji se pokreće paralelno. Elixir kôd se pokreće unutar Elixir procesora. Različiti delovi kôda se mogu pokrenuti u različitim procesima i mogu komunicirati međusobno. Veoma je tolerantan na greške. Koristi OTP (Open Telecom Platform) što je framework za Erlang koji jeziku omogućava da ukoliko se desi neka greška u vašem kôdu, samo taj proces stane i onda se restartuje. S obzirom na to da je kôd paralelizovan, ta greška neće uticati na ostale procese.

\section{Osnovna uputstva}
Vaš seminarski rad mora da sadrži najmanje jednu \textbf{sliku}, najmanje jednu \textbf{tabelu} i najmanje \textbf{sedam referenci} u spisku literature. Najmanje jedna slika treba da bude originalna i da predstavlja neke podatke koje ste Vi osmislili da treba da prezentujete u svom radu. Isto važi i za najmanje jednu tabelu. 	Od referenci, neophodno je imati bar jednu \textbf{knjigu}, bar jedan \textbf{naučni članak} iz odgovarajućeg časopisa i bar jednu adekvatnu \textbf{veb adresu}. 

\textbf{Dužina seminarskog rada treba da bude od 10 do 12 strana.} Svako prekoračenje ili potkoračenje biće kažnjeno sa odgovarajućim brojem poena. Eventualno, nakon strane 12, može se javiti samo tekst poglavlja \textbf{Dodatak} koji sadrži nekakav dodatni k\^{o}d, ali je svakako potrebno da rad može da se pročita i razume i bez čitanja tog dodatka. 

Ко жели, може да пише рад ћирилицом. У том случају, неопходно је да су инсталирани одговарајући пакети: texlive-fonts-extra, texlive-latex-extra, texlive-lang-cyrillic, texlive-lang-other. 

Nemojte koristiti stari način pisanja slova, tj ovo:
\begin{verbatim}
\v{s} i \v{c} i \'c ...
\end{verbatim}
Koristite direknto naša slova:	
\begin{verbatim}
š i č i ć ... 
\end{verbatim}


\section{Engleski termini i citiranje}	
\label{sec:termini_i_citiranje}

Na svakom mestu u tekstu naglasiti odakle tačno potiču informacije. Uz sve novouvedene termine u zagradi naglasiti od koje engleske reči termin potiče. 

Naredni primeri ilustruju način uvođenja enlegskih termina kao i citiranje.

\begin{primer}
Problem zaustavljanja (eng.~{\em halting problem}) je neodlučiv \cite{haltingproblem}.
\end{primer}

\begin{primer}
Za prevođenje programa napisanih u programskom jeziku C može se koristiti GCC kompajler \cite{gcc}.
\end{primer}

\begin{primer}
 Da bi se ispitivala ispravost softvera, najpre je potrebno precizno definisati njegovo ponašanje \cite{laski2009software}. 
\end{primer}

Reference koje se koriste u ovom tekstu zadate su u datoteci {\em seminarski.bib}. Prevođenje u pdf format u Linux okruženju može se uraditi na sledeći način:
\begin{verbatim}
pdflatex TemaImePrezime.tex 
bibtex TemaImePrezime.aux 
pdflatex TemaImePrezime.tex 
pdflatex TemaImePrezime.tex 
\end{verbatim}
Prvo latexovanje je neophodno da bi se generisao {\em .aux} fajl. {\em bibtex} proizvodi odgovarajući {\em .bbl} fajl koji se koristi za generisanje literature. 
Potrebna su dva prolaza (dva puta pdflatex) da bi se reference ubacile u tekst (tj da ne bi ostali znakovi pitanja umesto referenci). Dodavanjem novih referenci potrebno je ponoviti ceo postupak.  











Broj naslova i podnaslova je proizvoljan. Neophodni su samo Uvod i Zaključak. Na poglavlja unutar teksta referisati se po potrebi. 
\begin{primer}
U odeljku \ref{sec:naslov1} precizirani su osnovni pojmovi, dok su zaključci dati u odeljku \ref{sec:zakljucak}.
\end{primer}

Još jednom da napomenem da nema razloga da pišete:
\begin{verbatim}
\v{s} i \v{c} i \'c ...
\end{verbatim}
Možete koristiti srpska slova
\begin{verbatim}
š i č i ć ... 
\end{verbatim}



\section{Slike i tabele}
\label{slike_i_tabele}

Slike i tabele treba da budu u svom okruženju, sa odgovarajućim naslovima, obeležene labelom da koje omogućava referenciranje. 

\begin{primer} Ovako se ubacuje slika. Obratiti pažnju da je dodato i 
\begin{verbatim}
\usepackage{graphicx}
\end{verbatim}

\begin{figure}[h!]
\begin{center}
\includegraphics[scale=0.75]{panda.jpg}
\end{center}
\caption{Pande}
\label{fig:pande}
\end{figure}

Na svaku sliku neophodno je referisati se negde u tekstu. Na primer, na slici \ref{fig:pande} prikazane su pande. 
\end{primer}

\begin{primer} I tabele treba da budu u svom okruženju, i na njih je neophodno referisati se u tekstu. Na primer, u tabeli \ref{tab:tabela1} su prikazana različita poravnanja u tabelama.

\begin{table}[h!]
\begin{center}
\caption{Razlčita poravnanja u okviru iste tabele ne treba koristiti jer su nepregledna.}
\begin{tabular}{|c|l|r|} \hline
centralno poravnanje& levo poravnanje& desno poravnanje\\ \hline
a &b&c\\ \hline
d &e&f\\ \hline
\end{tabular}
\label{tab:tabela1}
\end{center}
\end{table}

\end{primer}

\section{K\^{o}d i paket listings}
Za ubacivanje koda koristite paket \textbf{listings}:
\url{https://en.wikibooks.org/wiki/LaTeX/Source_Code_Listings}

\begin{primer}
Primer ubacivanja koda za programski jezik Python dat je kroz listing \ref{simple}. Za neki drugi programski jezik, treba podesiti odgvarajući programski jezik u okviru defnisanja stila.
\end{primer}
\begin{lstlisting}[caption={Primer ubacivanja koda u tekst},frame=single, label=simple]
# This program adds up integers in the command line
import sys
try:
    total = sum(int(arg) for arg in sys.argv[1:])
    print 'sum =', total
except ValueError:
    print 'Please supply integer arguments'
\end{lstlisting}


\section{Prvi naslov}
\label{sec:naslov1}


Ovde pišem tekst. 
Ovde pišem tekst. 
Ovde pišem tekst. 
Ovde pišem tekst. 
Ovde pišem tekst. 
Ovde pišem tekst. 
Ovde pišem tekst. 
Ovde pišem tekst. 


\subsection{Prvi podnaslov}
\label{subsec:podnaslov1}

Ovde pišem tekst. 
Ovde pišem tekst. 
Ovde pišem tekst. 
Ovde pišem tekst. 
Ovde pišem tekst. 
Ovde pišem tekst. 
Ovde pišem tekst. 

\subsection{Drugi podnaslov}
\label{subsec:podnaslov2}

Ovde pišem tekst. 
Ovde pišem tekst. 
Ovde pišem tekst. 
Ovde pišem tekst. 
Ovde pišem tekst. 
Ovde pišem tekst. 


\subsection{... podnaslov}
\label{subsec:podnaslovN}

Ovde pišem tekst. 
Ovde pišem tekst. 
Ovde pišem tekst. 
Ovde pišem tekst. 
Ovde pišem tekst. 
Ovde pišem tekst. 

\section{n-ti naslov}
\label{sec:naslovN}

Ovde pišem tekst. 
Ovde pišem tekst. 
Ovde pišem tekst. 
Ovde pišem tekst. 
Ovde pišem tekst. 

\subsection{... podnaslov}
\label{subsec:podnaslovK}

Ovde pišem tekst. 
Ovde pišem tekst. 
Ovde pišem tekst. 
Ovde pišem tekst. 
Ovde pišem tekst. 

\subsection{... podnaslov}
\label{subsec:podnaslovM}

Ovde pišem tekst. 
Ovde pišem tekst. 
Ovde pišem tekst. 
Ovde pišem tekst. 
Ovde pišem tekst. 


\section{Zaključak}
\label{sec:zakljucak}

Ovde pišem zaključak. 
Ovde pišem zaključak. 
Ovde pišem zaključak. 
Ovde pišem zaključak. 
Ovde pišem zaključak. 
Ovde pišem zaključak. 
Ovde pišem zaključak. 
Ovde pišem zaključak. 
Ovde pišem zaključak. 
Ovde pišem zaključak. 
Ovde pišem zaključak. 
Ovde pišem zaključak. 


\addcontentsline{toc}{section}{Literatura}
\appendix
\bibliography{seminarski} 
\bibliographystyle{plain}

\appendix
\section{Dodatak}
Ovde pišem dodatne stvari, ukoliko za time ima potrebe.
HOvde pišem dodatne stvari, ukoliko za time ima potrebe.
Ovde pišem dodatne stvari, ukoliko za time ima potrebe.
Ovde pišem dodatne stvari, ukoliko za time ima potrebe.
Ovde pišem dodatne stvari, ukoliko za time ima potrebe.


\end{document}

